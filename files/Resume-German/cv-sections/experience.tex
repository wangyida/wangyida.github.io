%----------------------------------------------------------------------------------------
%	SECTION TITLE
%----------------------------------------------------------------------------------------

\cvsection{Erfahrung}

%----------------------------------------------------------------------------------------
%	SECTION CONTENT
%----------------------------------------------------------------------------------------

\begin{cventries}

%------------------------------------------------

\cventry
{Maschinenlernende Forscher} % Job title
{Pattern Recognition und Intelligent System Lab} % Organization
{Peking, China} % Location
{Sep. 2014 - Mar. 2017} % Date(s)
{ % Description(s) of tasks/responsibilities
\begin{cvitems}
\item {Mentor: \href{http://www.pris.net.cn/introduction/teacher/dengweihong}{Weihong Deng} (http://www.pris.net.cn/introduction/teacher/dengweihong). Forschung auf Gesichtserkennung, Objekterkennung, semantische Segmentierung auf der Grundlage traditioneller Methoden und tiefes Lernen.}
\end{cvitems}
}
%------------------------------------------------

\cventry
{Maschinelles Lernen Forscher und Software Engineer} % Job title
{Google Open Source \& OpenCV} % Organization
{Peking, China} % Location
{Apr. 2015 - Sep. 2016} % Date(s)
{ % Description(s) of tasks/responsibilities
\begin{cvitems}
\item {Mentor: \href{https://www.linkedin.com/in/stefano-fabri-16a73748}{Stefano Fabri} and \href{https://www.linkedin.com/in/manuele-tamburrano-b82384a5?authType=name&authToken=Di5p&trk=prof-sb-browse_map-name}{Manuele Tamburrano}. Unterstützt von Google Summer of Code 2015 bis 2016. Die Aufrechterhaltung tiny-dnn Projekt und die Entwicklung API 3D-Objekterkennung. Mitglied des Projektes tiny-dnn mit weiteren fünf Forschern. \\
Hier sind Online-Video-Demos für sie mit Hyperlinks: \href{https://www.youtube.com/watch?v=Mc20rTYdXTE}{3D Object Multi-task Learning} und \href{https://drive.google.com/open?id=0B-RYa1FDOrYXVUEzcG1mdnl5a3M}{tiny-dnn on iOS}
}
\end{cvitems}
}

%------------------------------------------------

\cventry
{Softwareentwickler} % Job title
{Alibaba} % Organization
{Peking, China} % Location
{Jun. 2015 - Jul. 2015} % Date(s)
{ % Description(s) of tasks/responsibilities
\begin{cvitems}
\item {Mitbewerber in Tianchi Big Data Contest 2015. Vorhersage der Kunden-Intention, Ranking der 68. Mannschaft in über 1500 Teams aus der ganzen Welt.}
\end{cvitems}
}

%------------------------------------------------

\cventry
{Softwareentwickler} % Job title
{WINE} % Organization
{Peking, China} % Location
{May. 2015 - Jun. 2015} % Date(s)
{ % Description(s) of tasks/responsibilities
\begin{cvitems}
\item {Charakter Matching, Extrahieren von Feature von PCANet, um Zahlen in OPEN SOURCE Zahlen mit Standard-Microsoft-Zahlen entsprechen.}
\end{cvitems}
}

%------------------------------------------------

\end{cventries}
